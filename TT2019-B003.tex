\documentclass[12pt, a4paper, titlepage]{article}
\usepackage[spanish]{babel}
\usepackage[utf8]{inputenc}
\usepackage{graphicx}
\usepackage{xcolor}

\definecolor{guindapoli}{RGB}{102, 0, 51}
\definecolor{azulescom}{RGB}{0, 0, 102}

\begin{document}
	
	%PORTADA
	\begin{titlepage}
		%\vspace*{} Para dejar un cierto espacio entre dos líneas.
		
		\vspace*{-1.5in}
		\begin{figure}[htb]
			\begin{center}
				\includegraphics[width=4cm]{./imagenes/logoipn.png}
			\end{center}
		\end{figure}
		
		\begin{center}
		\begin{LARGE}
			\textcolor{guindapoli}{INSTITUTO POLITÉCNICO NACIONAL}\\
		\end{LARGE}	
		\vspace*{0.2in}
		\begin{Large}
			\textcolor{azulescom}{ESCUELA SUPERIOR DE CÓMPUTO}\\
		\end{Large}		
		\vspace*{0.4in}
		\begin{large}
			Trabajo Terminal I.\\
		\end{large}
		\vspace*{0.2in}
		\begin{Large}
			\textbf{Autentificación Mediante Chaffing And Winnowing En El Protocolo HTTP}\\2019-B003\\
		\end{Large}
		\vspace*{0.05in}
		\rule{80mm}{0.1mm}\\
		\vspace*{0.1in}
		\begin{large}
			\begin{center}
				Integrantes:\\
				Blancas Pérez Bryan Israel\\
				Carrillo Fernández Gerardo\\
				Morales González Diego Arturo\\
				Paredes Hernández Pedro Antonio\\
			\end{center}
		\end{large}
		\begin{large}
			Directores:\\
			Moreno Cervantes Axel Ernesto\\
			Díaz Santiago Sandra\\
		\end{large}
		\end{center}
	\end{titlepage}

	\begin{appendix}
		%%Índice
		\renewcommand*\contentsname{Índice}
		\tableofcontents
		\newpage
		%%índice de figuras
		\listoffigures
		\newpage
		%%Índice de tablas
		\newpage
		\listoftables
	\end{appendix}
	\newpage
	

	\section{Introducción.}
		\subsection{Planteamiento del problema.}
		\subsection{Justificación.}
		\subsection{Objetivos.}
		\subsection{Metodología.}
		\subsection{Estado del Arte.}
	\newpage
	\section{Marco Teórico.}
		\subsection{Formato a decidir.}
	\newpage
	\section{Análisis}
		\subsection{Prototipo I.}
			\subsubsection{Herramientas a usar.}
			\subsubsection{Estudio de requerimientos.}
			\subsubsection{Reglas del negocio.}
	\newpage
	\section{Desarrollo}
		\subsection{Prototipo I.}
			\subsubsection{Diagrama de casos de uso.}
			\subsubsection{Descripción de casos de uso.}
			\subsubsection{Diagrama de flujo.}
			\subsubsection{Flujo de datos.}
			\subsubsection{Diagrama de clases.}
			\subsubsection{Diagrama de secuencia.}
			\subsubsection{Interfaz de usuario.}
			\subsubsection{Requisitos de diseño.}
			
\end{document}
